\documentclass[draft,12pt]{report}

\usepackage{hyperref}
\hypersetup{
  colorlinks,
  final,
  citecolor=black,
  filecolor=black,
  linkcolor=black,
  urlcolor=black
}


\title{Evaluating the Resource Management Possibilites of the Vulkan Graphics API}
\date{\today}
\author{Manuel Maier}


\begin{document}
  %
  % Title Page
  %
  \pagenumbering{gobble}
  \maketitle
  \newpage

  %
  % Table of Contents
  %
  \pagenumbering{Roman}
  \tableofcontents
  \newpage


  %
  % Chapters
  %
  \pagenumbering{arabic}


  \chapter{Introduction}
    General introduction to computer graphics. One of the biggest problems: Performance. Especially with virtual reality.

    Maybe mention the author's (that's me!) bias towards game development.

    \section{High Level Graphics Workflow}
      General overview of the stages several resources (vertices, textures, etc.) have to go through

    \section{Motivation for a new Graphics API}
      Basically write why Vulkan is needed.

    \section{The Vulkan Graphics System}
      What does it do. Where does it come from. What are people expecting from it. Cross-platform nature (in comparison maybe to PS4's libGNM made specifically for PS4 hardware).

      \subsection{Vulkan Competitors}
        OpenGL, Direct3D11, Direct3D12, libGNM (PS4)

    \section{Document Structure}
      The structure and content of this document.

  \chapter{Vulkan API Overview/Workflow}
    Vulkan workflow, object creation, patterns (vkCreate/vkDestroy, vkAllocate/vkDeallocate), synchronization.

    Listing and short explanation of Vulkan components such as buffers, images, command buffers.

    Note that when talking about a "device", a logical device is meant. When referring to the actual hardware component the term "physical device" will be used.

    \section{Layers and Extensions}
      Layers are instance-only, extensions are both on the isntance-level as well as the device-level.

    \section{Memory Management}
      Allocate memory separately and then binding it to resources. Decoupled on purpose for custom management.

    \section{Shader Language: SPIR-V}
      SPIR = "Standard Portable Intermediate Representation".

    \section{Synchronisation Primitives}
      Describe what Vulkan offers and discuss the individual use-cases as well as their performance impact.

    \section{Interaction with the Operating System}
      Windowing system, specific Vulkan extensions per OS.


  \chapter{GPU Resources}
    Different kinds of resources for different kinds of operations. These have been introduced in the previous chapter already and will now be explained in more detail.

    Possible pattern per section:
    \begin{itemize}
      \item What is the resource used for
      \item How is it allocated
      \item Discuss synchronisation issues.
    \end{itemize}

    \section{Images and Buffers}
      Difference and why it matters.

    \section{Command Buffers}
      Record commands into buffers that are uploaded to the gpu to be replayed there.

    \section{Shaders and Pipelines}
      Data flow between CPU and GPU. Data flow within the GPU through shaders.

      Descriptor pool/layout/set/set-layout


  \chapter{Resource Management Techniques}
    Discuss what to do in certain situations, e.g. when uploading image data using a staging buffer.


  \chapter{Evaluation}
    Present and discuss preformance measurement results.


  \chapter{Conclusion}
    API quality. Explain and discuss why the API is called "modern", "highly efficient", and "explicit".

\end{document}

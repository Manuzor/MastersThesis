%!TEX root = ../Main.tex

\ifthenelse{\boolean{hdm}}
{
\chapter*{Kurzfassung}
\label{Abstract_de}
  Dies ist die Kurzfassung.

  \lipsum[1-2]

  \newpage
}
{}

\chapter*{Abstract}
\label{Abstract}

% \todo[inline]
% {
%   New Hardware $\rightarrow$ new APIs.

%   More control of the hardware.

%   Setup of Vulkan dev env.

%   walkthrough of rendering.

%   In the last years graphics hardware changed drastically becoming more and more generalized.
%   [...]
%   Traditional graphics APIs were build with the old systems in mind, not considering things like multi-threading, and don't fit the new technological designs, which has massive impact on performance.
%   [...]
%   From this the need for newly designed graphics APIs arose.
%   [...]
%   One of these is the graphics API Vulkan.
%   This work gives an overview of the Vulkan graphics API, presents general instructions on how to set up a development environment, and finally analyzes the major features of Vulkan using prototypical programming examples showing the use of the API itself.
%   [...]
% }

In recent years, consumer graphics hardware has changed considerably.
In the past, the hardware consisted only of fixed-function components that performed the necessary computations in a massively parallel manner to transform three-dimensional data to an image displayed on a screen.
At the same time, APIs to graphics hardware were developed.
As hardware design changed over the years, mainly to improve on performance, it has also become more generalized and accessible to other kinds of computational tasks\todo{Add ``than computer graphics''?}.
Traditional APIs had to adjust to these design changes while maintaining compatibility to previously supported designs at the same time.
This led to an increase in driver complexity ultimately impacting performance in a negative way.
New graphics APIs were designed to match recent hardware more closely, one of which is the Vulkan graphics API.

In this work, the Vulkan graphics API is presented in three stages.
At first, an overview of the API itself is provided.
Afterwards, general instructions on how to set up an environment for developing Vulkan applications are given.
And finally, the fundamentals of rendering an image using Vulkan are presented with the aid of simplified, prototypical examples.

%!TEX root = ../Main.tex

\chapter{Conclusion}
\label{cha:Conclusion}
  \todo[inline]{API quality. Explain and discuss why the API is called `modern', `highly efficient', and `explicit'.}
  \todo[inline]{Highly active development. LunarG provides more and more improved layers with each release.}
  \todo[inline]{Driver's are slow to keep up. No windows support from Intel for now.}
  \todo[inline]{Hard to learn, but probably worth it. Validation layers are awesome!}
  \todo[inline]{Mention Valve's plans for pipeline caches? Download pipeline caches when installing games.}
  \tbd

  In this chapter, I am giving some insight into my thoughts about Vulkan, the problems I had with it, as well as the future I see for it.

  With rising driver complexity and OpenGL expanding to support new interfaces, becoming more confusing from an \gls{api} usage point of view, seeing Vulkan as a cross-platform successor to OpenGL is good.\todo{Rephrase.}{}
  Vulkan is created by industry professionals who work closely with related technology all the time.
  Providing more control to the application is generally good but it also increases complexity of the application itself.
  This might not be feasible for some applications.
  Another aspect is driver support for Vulkan.
  The prototype referred to in appendix~\ref{apx:PrototypeSource} was developed on a system that was specfically acquired for the purpose of developing a Vulkan application because the system I owned previously mounted an NVIDIA GeForce GTX 560Ti, which is not supported by NVIDIA drivers to this date.

  The development of the prototype

  Vulkan is a modern api.
  It supports desktop and mobile platforms.
  Development has improved drastically since release.

  Future of Vulkan.
    - Multi-GPU
    - Pipeline cache: grab from server when installing a game.

  Khronos is very actively involving the community.
  Many things open-source on GitHub.

  An exciting feature about Vulkan is the pipeline cache.
  Creating a graphics pipeline in Vulkan is quite expensive because it does optimizations do the shaders in use and to the entire pipeline as a whole.
  The pipeline cache can be used to speed up this process.
  In the Vulkan session part II at GDC 2016\cite{vksessiongdc16} Dan Ginsburg, one of the speakers at this talk, described a scenario where the online game distribution platform \textit{steam} could be used to pull pipeline caches from a server for a particular game at the time it is installed.
  This would improve performance of that game at the time it is run for the first time.

  The Vulkan specification~1.0 is only the beginning.
  It is missing features like support for copying device memory from one GPU directly to another without involving the hosti as well as an API for high performance and high-quality 2D rendering, like Direct2D.
  It remains to be seen how Vulkan, and the industry around it, will evolve in the future.

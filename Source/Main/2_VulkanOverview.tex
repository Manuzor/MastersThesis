%!TEX root = ../Main.tex

\chapter{Vulkan API Overview/Workflow}
\label{cha:VulkanOverview}
  \todo[inline]
  {
    The Vulkan Graphics \acrfull{api} workflow, object creation, patterns (vkCreate/vkDestroy, vkAllocate/vkDeallocate), synchronization.

    Listing and short explanation of Vulkan components such as buffers, images, command buffers.

    Note that when talking about a "device", a logical device is meant. When referring to the actual hardware component the term "physical device" will be used.
  }

  \lipsum

  \section{Layers and Extensions}
    \todo[inline]{Layers are instance-only, extensions are both on the isntance-level as well as the device-level.}

    \lipsum

  \section{Memory Management}
    \todo[inline]{Allocate memory separately and then binding it to resources. Decoupled on purpose for custom management.}

    \lipsum

  \section{Shader Language: SPIR-V}
    \todo[inline]{\acrfull{spirv}}

    \lipsum

  \section{Synchronisation Primitives}
    \todo[inline]{Describe what Vulkan offers and discuss the individual use-cases as well as their performance impact.}

    \lipsum

  \section{Interaction with the Operating System}
    \todo[inline]{Windowing system, specific Vulkan extensions per OS.}

    \lipsum
